% Options for packages loaded elsewhere
\PassOptionsToPackage{unicode}{hyperref}
\PassOptionsToPackage{hyphens}{url}
%
\documentclass[
]{article}
\usepackage{lmodern}
\usepackage{amssymb,amsmath}
\usepackage{ifxetex,ifluatex}
\ifnum 0\ifxetex 1\fi\ifluatex 1\fi=0 % if pdftex
  \usepackage[T1]{fontenc}
  \usepackage[utf8]{inputenc}
  \usepackage{textcomp} % provide euro and other symbols
\else % if luatex or xetex
  \usepackage{unicode-math}
  \defaultfontfeatures{Scale=MatchLowercase}
  \defaultfontfeatures[\rmfamily]{Ligatures=TeX,Scale=1}
\fi
% Use upquote if available, for straight quotes in verbatim environments
\IfFileExists{upquote.sty}{\usepackage{upquote}}{}
\IfFileExists{microtype.sty}{% use microtype if available
  \usepackage[]{microtype}
  \UseMicrotypeSet[protrusion]{basicmath} % disable protrusion for tt fonts
}{}
\makeatletter
\@ifundefined{KOMAClassName}{% if non-KOMA class
  \IfFileExists{parskip.sty}{%
    \usepackage{parskip}
  }{% else
    \setlength{\parindent}{0pt}
    \setlength{\parskip}{6pt plus 2pt minus 1pt}}
}{% if KOMA class
  \KOMAoptions{parskip=half}}
\makeatother
\usepackage{xcolor}
\IfFileExists{xurl.sty}{\usepackage{xurl}}{} % add URL line breaks if available
\IfFileExists{bookmark.sty}{\usepackage{bookmark}}{\usepackage{hyperref}}
\hypersetup{
  pdftitle={Exam Paper - Backtester Project},
  pdfauthor={Michelangelo De Francesco},
  hidelinks,
  pdfcreator={LaTeX via pandoc}}
\urlstyle{same} % disable monospaced font for URLs
\usepackage[margin=1in]{geometry}
\usepackage{color}
\usepackage{fancyvrb}
\newcommand{\VerbBar}{|}
\newcommand{\VERB}{\Verb[commandchars=\\\{\}]}
\DefineVerbatimEnvironment{Highlighting}{Verbatim}{commandchars=\\\{\}}
% Add ',fontsize=\small' for more characters per line
\usepackage{framed}
\definecolor{shadecolor}{RGB}{248,248,248}
\newenvironment{Shaded}{\begin{snugshade}}{\end{snugshade}}
\newcommand{\AlertTok}[1]{\textcolor[rgb]{0.94,0.16,0.16}{#1}}
\newcommand{\AnnotationTok}[1]{\textcolor[rgb]{0.56,0.35,0.01}{\textbf{\textit{#1}}}}
\newcommand{\AttributeTok}[1]{\textcolor[rgb]{0.77,0.63,0.00}{#1}}
\newcommand{\BaseNTok}[1]{\textcolor[rgb]{0.00,0.00,0.81}{#1}}
\newcommand{\BuiltInTok}[1]{#1}
\newcommand{\CharTok}[1]{\textcolor[rgb]{0.31,0.60,0.02}{#1}}
\newcommand{\CommentTok}[1]{\textcolor[rgb]{0.56,0.35,0.01}{\textit{#1}}}
\newcommand{\CommentVarTok}[1]{\textcolor[rgb]{0.56,0.35,0.01}{\textbf{\textit{#1}}}}
\newcommand{\ConstantTok}[1]{\textcolor[rgb]{0.00,0.00,0.00}{#1}}
\newcommand{\ControlFlowTok}[1]{\textcolor[rgb]{0.13,0.29,0.53}{\textbf{#1}}}
\newcommand{\DataTypeTok}[1]{\textcolor[rgb]{0.13,0.29,0.53}{#1}}
\newcommand{\DecValTok}[1]{\textcolor[rgb]{0.00,0.00,0.81}{#1}}
\newcommand{\DocumentationTok}[1]{\textcolor[rgb]{0.56,0.35,0.01}{\textbf{\textit{#1}}}}
\newcommand{\ErrorTok}[1]{\textcolor[rgb]{0.64,0.00,0.00}{\textbf{#1}}}
\newcommand{\ExtensionTok}[1]{#1}
\newcommand{\FloatTok}[1]{\textcolor[rgb]{0.00,0.00,0.81}{#1}}
\newcommand{\FunctionTok}[1]{\textcolor[rgb]{0.00,0.00,0.00}{#1}}
\newcommand{\ImportTok}[1]{#1}
\newcommand{\InformationTok}[1]{\textcolor[rgb]{0.56,0.35,0.01}{\textbf{\textit{#1}}}}
\newcommand{\KeywordTok}[1]{\textcolor[rgb]{0.13,0.29,0.53}{\textbf{#1}}}
\newcommand{\NormalTok}[1]{#1}
\newcommand{\OperatorTok}[1]{\textcolor[rgb]{0.81,0.36,0.00}{\textbf{#1}}}
\newcommand{\OtherTok}[1]{\textcolor[rgb]{0.56,0.35,0.01}{#1}}
\newcommand{\PreprocessorTok}[1]{\textcolor[rgb]{0.56,0.35,0.01}{\textit{#1}}}
\newcommand{\RegionMarkerTok}[1]{#1}
\newcommand{\SpecialCharTok}[1]{\textcolor[rgb]{0.00,0.00,0.00}{#1}}
\newcommand{\SpecialStringTok}[1]{\textcolor[rgb]{0.31,0.60,0.02}{#1}}
\newcommand{\StringTok}[1]{\textcolor[rgb]{0.31,0.60,0.02}{#1}}
\newcommand{\VariableTok}[1]{\textcolor[rgb]{0.00,0.00,0.00}{#1}}
\newcommand{\VerbatimStringTok}[1]{\textcolor[rgb]{0.31,0.60,0.02}{#1}}
\newcommand{\WarningTok}[1]{\textcolor[rgb]{0.56,0.35,0.01}{\textbf{\textit{#1}}}}
\usepackage{graphicx,grffile}
\makeatletter
\def\maxwidth{\ifdim\Gin@nat@width>\linewidth\linewidth\else\Gin@nat@width\fi}
\def\maxheight{\ifdim\Gin@nat@height>\textheight\textheight\else\Gin@nat@height\fi}
\makeatother
% Scale images if necessary, so that they will not overflow the page
% margins by default, and it is still possible to overwrite the defaults
% using explicit options in \includegraphics[width, height, ...]{}
\setkeys{Gin}{width=\maxwidth,height=\maxheight,keepaspectratio}
% Set default figure placement to htbp
\makeatletter
\def\fps@figure{htbp}
\makeatother
\setlength{\emergencystretch}{3em} % prevent overfull lines
\providecommand{\tightlist}{%
  \setlength{\itemsep}{0pt}\setlength{\parskip}{0pt}}
\setcounter{secnumdepth}{-\maxdimen} % remove section numbering

\title{Exam Paper - Backtester Project}
\author{Michelangelo De Francesco}
\date{22/3/2021}

\begin{document}
\maketitle

{
\setcounter{tocdepth}{2}
\tableofcontents
}
\hypertarget{objective}{%
\section{Objective}\label{objective}}

The objective of the project is to create a simple \textbf{backtester}
that is able to test the performance of trading systems, given as an
input a \texttt{CSV} downloaded from
\href{https://www.investing.com/}{Investing USA}.

\hypertarget{the-project-in-practice}{%
\subsection{The project, in Practice}\label{the-project-in-practice}}

In this project, a simple backtesting engine has been created and a
simple strategy based on
\href{https://www.investopedia.com/terms/r/rsi.asp}{Relative Strength
Indicator} has been used as a practical example to show the
effectiveness of the backtester.

Regarding the RSI Strategy, it relies on simple Technical Analysis and
on the concept of over-sold and over-bought: the strategy will stay in
\texttt{IDLE} when the indicator is between the upper and the lower
boundaries, while it will recommend to \texttt{BUY} or \texttt{SELL}
when it will be respectively below the lower level and above the upper
one.

\hypertarget{notes-on-the-project}{%
\subsection{Notes on the Project}\label{notes-on-the-project}}

\begin{enumerate}
\def\labelenumi{\arabic{enumi}.}
\tightlist
\item
  It's written in vanilla R language, external packages has not been
  used as I aimed to code everything from scratch instead of using
  pre-built functions
\item
  I have made extensive use of Object Oriented Programming in R Language
  (i.e.~\texttt{Indicators}, \texttt{Portfolio} and \texttt{Strategies}
  are Objects, while \texttt{Orders} are rows in a \texttt{data.frame}
  due to motivations that will be explained in the following sections)
\end{enumerate}

\hypertarget{architecture-description}{%
\section{Architecture Description}\label{architecture-description}}

The Architectural design behind this project is based on Object Oriented
Programming, therefore, as already said, some components of the project
are Objects created by their respective Classes.

The main reason behind this choice is the need for entities from which I
could be able to call attributes (variables associated to the entity)
and methods. For instance, regarding the portfolio, I wanted to have an
entity that represented the portfolio itself and from which I could have
the value of it, but also functions such as \texttt{addOrder} or
\texttt{closeOrder}.

Object Oriented Programming is substantiated by the creation of classes,
which in are are declared using the function \texttt{setRefClass()}
which creates an S4 class (an updated and more complete version of the
S3 classes as it provides the possibility to create custom functions and
use the object itself inside the function).

Thanks to the simple reproducibility of objects due to classes (in fact,
they can be considered as blueprints of objects), I have decided to
adopt this data type also for \texttt{Indicators} and
\texttt{Strategies} for scalability reasons and for the motivations
already stated.

\hypertarget{note-on-architectural-decision-on-order-entity}{%
\subsection{Note on architectural decision on Order
Entity}\label{note-on-architectural-decision-on-order-entity}}

The \texttt{Order} Entity, which represents a buy or sell order, is not
an object although it would be eligible. The motivation is that in R
Language, Objects can be stored in vectors, but they instantly lose
their parameters and methods as shown below.

\begin{Shaded}
\begin{Highlighting}[]
\NormalTok{gen <-}\StringTok{ }\KeywordTok{setRefClass}\NormalTok{(}\StringTok{"myRefClass"}\NormalTok{, }
                   \DataTypeTok{fields=}\KeywordTok{list}\NormalTok{(}\DataTypeTok{aa=}\StringTok{"character"}\NormalTok{),}
                   \DataTypeTok{methods=}\KeywordTok{list}\NormalTok{(}
                     \DataTypeTok{hi=} \ControlFlowTok{function}\NormalTok{()\{}
                       \KeywordTok{cat}\NormalTok{(}\StringTok{"hi"}\NormalTok{)}
\NormalTok{                     \}}
\NormalTok{                   )}
\NormalTok{)}
\NormalTok{x <-}\StringTok{ }\NormalTok{gen}\OperatorTok{$}\KeywordTok{new}\NormalTok{(}\DataTypeTok{aa=}\StringTok{"nome"}\NormalTok{)}
\NormalTok{y <-}\StringTok{ }\NormalTok{gen}\OperatorTok{$}\KeywordTok{new}\NormalTok{(}\DataTypeTok{aa=}\StringTok{"bana"}\NormalTok{)}
\NormalTok{a<-}\KeywordTok{c}\NormalTok{(x,y)}

\NormalTok{b<-a[}\DecValTok{1}\NormalTok{]}
\NormalTok{b}\OperatorTok{$}\NormalTok{aa }\CommentTok{# NULL}
\end{Highlighting}
\end{Shaded}

\begin{verbatim}
## NULL
\end{verbatim}

\begin{Shaded}
\begin{Highlighting}[]
\CommentTok{#b$hi() throws an error}
\NormalTok{a[}\DecValTok{1}\NormalTok{]}\OperatorTok{$}\NormalTok{aa }\CommentTok{# NULL}
\end{Highlighting}
\end{Shaded}

\begin{verbatim}
## NULL
\end{verbatim}

Nonetheless, the main idea was to store all the orders inside a vector
of the portfolio. In order to work around the problem, it has been
treated as a simple row in a data.frame.

\hypertarget{how-to-start-the-project}{%
\section{How to Start the Project}\label{how-to-start-the-project}}

The project is already set up and it's possible to run it without any
change: run the \texttt{init.R} code line by line or as a whole.

If, instead, it is needed to change the input parameters, it is
mandatory to:

\begin{enumerate}
\def\labelenumi{\arabic{enumi}.}
\tightlist
\item
  download historical data from
  \href{https://www.investing.com/}{Investing USA} in CSV, such as
  \href{https://www.investing.com/equities/apple-computer-inc-historical-data}{Apple
  Stock}
\item
  change in the file \texttt{init.R} the input parameters
  a.\texttt{pathOfFinancialData}, which is the path of the
  \texttt{.CSV}. Note: you can use both relative and absolute path.
\end{enumerate}

\begin{enumerate}
\def\labelenumi{\alph{enumi}.}
\setcounter{enumi}{1}
\tightlist
\item
  \texttt{initialPortfolioValue}, which is the amount of money in the
  portfolio at the beginning of the simulation.
\item
  \texttt{percentageOfPortfolioForEachInvestment}, which is the
  percentage of the portfolio which is invested in each trade.
\item
  \texttt{indicatorsParameters}: the parameters chosen for the strategy,
  that are in this case respectively: the periods of the RSI, the lower
  bound and the upper 60
\end{enumerate}

An example is provided below:

\begin{Shaded}
\begin{Highlighting}[]
\NormalTok{path =}\StringTok{ ".}\CharTok{\textbackslash{}\textbackslash{}}\StringTok{inputCSVData}\CharTok{\textbackslash{}\textbackslash{}}\StringTok{S&P 500 Historical Data Reversed with missing data.csv"}
\CommentTok{# or using absolute path}
\CommentTok{# path = "C:\textbackslash{}\textbackslash{}Users\textbackslash{}\textbackslash{}Michelangelo\textbackslash{}\textbackslash{}Desktop\textbackslash{}\textbackslash{}Projects\textbackslash{}\textbackslash{}BacktesteR-MVP\textbackslash{}\textbackslash{}inputCSVData\textbackslash{}\textbackslash{}S&P 500 Historical Data Reversed with missing data.csv"}
\NormalTok{initialPortfolioValue =}\StringTok{ }\DecValTok{10000}
\NormalTok{percentageOfPortfolioForEachInvestment=}\FloatTok{0.1}

\NormalTok{indicatorsParameters=}\StringTok{ }\KeywordTok{c}\NormalTok{(}\DecValTok{8}\NormalTok{, }\DecValTok{40}\NormalTok{, }\DecValTok{60}\NormalTok{)}
\end{Highlighting}
\end{Shaded}

After having the parameters set, start the code in \texttt{init.R}
line-by-line or as a whole.

\hypertarget{backtester-description}{%
\section{Backtester Description}\label{backtester-description}}

In this section, it will be shown how the code works under the hood and
will be explained the logical steps of the backtester starting from the
first line of \texttt{init.R}.

\hypertarget{initial-setup}{%
\subsection{Initial Setup}\label{initial-setup}}

A new RSI Strategy Object is created from the \texttt{singleRSIStrategy}
Class after setting the environment and the input paramenters.

It is important to not that the English language using
\texttt{Sys.setlocale(locale\ =\ "English")} in order to parse the dates
present inside the CSV.

\begin{Shaded}
\begin{Highlighting}[]
\KeywordTok{rm}\NormalTok{(}\DataTypeTok{list=}\KeywordTok{ls}\NormalTok{())}
\KeywordTok{graphics.off}\NormalTok{()}
\KeywordTok{Sys.setlocale}\NormalTok{(}\DataTypeTok{locale =} \StringTok{"English"}\NormalTok{)}
\end{Highlighting}
\end{Shaded}

\begin{verbatim}
## [1] "LC_COLLATE=English_United States.1252;LC_CTYPE=English_United States.1252;LC_MONETARY=English_United States.1252;LC_NUMERIC=C;LC_TIME=English_United States.1252"
\end{verbatim}

\begin{Shaded}
\begin{Highlighting}[]
\CommentTok{# Input options}
\NormalTok{pathOfFinancialData =}\StringTok{ ".}\CharTok{\textbackslash{}\textbackslash{}}\StringTok{inputCSVData}\CharTok{\textbackslash{}\textbackslash{}}\StringTok{S&P 500 Historical Data Reversed with missing data.csv"}
\NormalTok{initialPortfolioValue =}\StringTok{ }\DecValTok{10000}
\NormalTok{percentageOfPortfolioForEachInvestment=}\FloatTok{0.1}
\NormalTok{indicatorsParameters=}\StringTok{ }\KeywordTok{c}\NormalTok{(}\DecValTok{8}\NormalTok{, }\DecValTok{40}\NormalTok{, }\DecValTok{60}\NormalTok{)}

\CommentTok{# Add paths}
\KeywordTok{source}\NormalTok{(}\StringTok{"dataInputFunctions.R"}\NormalTok{)}
\KeywordTok{source}\NormalTok{(}\StringTok{"portfolioModel.R"}\NormalTok{)}
\KeywordTok{source}\NormalTok{(}\StringTok{"singleRSIStrategyModel.R"}\NormalTok{)}
\KeywordTok{source}\NormalTok{(}\StringTok{"orderFunctions.R"}\NormalTok{)}
\KeywordTok{source}\NormalTok{(}\StringTok{"backtest.R"}\NormalTok{)}
\KeywordTok{source}\NormalTok{(}\StringTok{"analytics.R"}\NormalTok{)}
\KeywordTok{source}\NormalTok{(}\StringTok{"./indicators/RSIModel.R"}\NormalTok{)}

\NormalTok{strategy =}\StringTok{ }\NormalTok{singleRSIStrategy}\OperatorTok{$}\KeywordTok{new}\NormalTok{(); }
\end{Highlighting}
\end{Shaded}

Then, the process of data retrieving and cleaning starts: 1. The CSV
parsed and stored into a \texttt{data.frame} 2. The data is then cleaned
from missing values( if in the row is present a \texttt{NA} value, the
row is removed) and is sorted by date 3. Finally, the data is indexed

\begin{Shaded}
\begin{Highlighting}[]
\NormalTok{rawData =}\StringTok{ }\KeywordTok{readDataFromCSV}\NormalTok{(pathOfFinancialData)}
\NormalTok{dataOutput =}\StringTok{ }\KeywordTok{cleanAndValidateData}\NormalTok{(rawData)}
\NormalTok{cleanData =}\StringTok{ }\NormalTok{dataOutput}\OperatorTok{$}\NormalTok{data}
\NormalTok{missingValues =}\StringTok{ }\NormalTok{dataOutput}\OperatorTok{$}\NormalTok{missingValues}
\NormalTok{missingValues}
\end{Highlighting}
\end{Shaded}

\begin{verbatim}
##   Date  Price Change 
##      0      3      0
\end{verbatim}

Moreover, the Portfolio is created with its initial values; valueHistory
is a \texttt{data.frame} where the chages of the portfolio value are
stored.

\begin{Shaded}
\begin{Highlighting}[]
\NormalTok{portfolio =}\StringTok{ }\NormalTok{portfolioClassGenerator}\OperatorTok{$}\KeywordTok{new}\NormalTok{(}
  \DataTypeTok{value=}\NormalTok{initialPortfolioValue, }
  \DataTypeTok{orders =} \KeywordTok{data.frame}\NormalTok{(),}
  \DataTypeTok{valueHistory =} \KeywordTok{data.frame}\NormalTok{(}
    \DataTypeTok{Date=}\KeywordTok{c}\NormalTok{(cleanData}\OperatorTok{$}\NormalTok{Date[}\DecValTok{1}\NormalTok{]),}
    \DataTypeTok{Value=}\KeywordTok{c}\NormalTok{(initialPortfolioValue),}
    \DataTypeTok{Change=}\KeywordTok{c}\NormalTok{(}\OtherTok{NA}\NormalTok{)}
\NormalTok{    )}
\NormalTok{  );}
\end{Highlighting}
\end{Shaded}

The last step before starting the backtest consists in adding the
parameters of the RSI to the strategy, in fact, the objective of the
function \texttt{addIndicators(values)} is to create a RSI object (with
input parameters the ones that has been chosen before) as a variable
inside the strategy and use it to produce signals.

\begin{Shaded}
\begin{Highlighting}[]
\NormalTok{strategy}\OperatorTok{$}\KeywordTok{addIndicators}\NormalTok{(indicatorsParameters)}
\end{Highlighting}
\end{Shaded}

\hypertarget{backtest}{%
\subsection{Backtest}\label{backtest}}

The backtest starts and the portfolio is returned. In this phase, orders
and signals are computed as it will be explained in the following
sections.

\begin{Shaded}
\begin{Highlighting}[]
\NormalTok{portfolio =}\StringTok{ }\KeywordTok{backtestStrategy}\NormalTok{(portfolio, strategy, cleanData)}
\end{Highlighting}
\end{Shaded}

\hypertarget{output-and-analytics}{%
\subsection{Output and Analytics}\label{output-and-analytics}}

In this section, a series of output, metrics and analytics is provided

Plot of price and returns of the timeseries analyzed:
\includegraphics{exam-paper_files/figure-latex/unnamed-chunk-3-1.pdf}
\includegraphics{exam-paper_files/figure-latex/unnamed-chunk-3-2.pdf}
RSI Indicator:
\includegraphics{exam-paper_files/figure-latex/unnamed-chunk-4-1.pdf}

Portfolio value at the end of the backtest, interest and standard
deviation:

\begin{Shaded}
\begin{Highlighting}[]
\NormalTok{portfolio}\OperatorTok{$}\NormalTok{value}
\end{Highlighting}
\end{Shaded}

\begin{verbatim}
## [1] 10009.79
\end{verbatim}

\begin{Shaded}
\begin{Highlighting}[]
\NormalTok{interest =}\StringTok{ }\NormalTok{(portfolio}\OperatorTok{$}\NormalTok{value}\OperatorTok{/}\NormalTok{initialPortfolioValue}\DecValTok{-1}\NormalTok{)}
\NormalTok{interest}
\end{Highlighting}
\end{Shaded}

\begin{verbatim}
## [1] 0.0009791586
\end{verbatim}

\begin{Shaded}
\begin{Highlighting}[]
\NormalTok{standardDeviation =}\StringTok{ }\KeywordTok{sd}\NormalTok{(portfolio}\OperatorTok{$}\NormalTok{valueHistory}\OperatorTok{$}\NormalTok{Value)}
\NormalTok{standardDeviation}
\end{Highlighting}
\end{Shaded}

\begin{verbatim}
## [1] 15.10555
\end{verbatim}

Changes in portfolio value:

\begin{Shaded}
\begin{Highlighting}[]
\NormalTok{portfolio}\OperatorTok{$}\NormalTok{valueHistory}
\end{Highlighting}
\end{Shaded}

\begin{verbatim}
##          Date     Value        Change
## 1  2020-07-30 10000.000            NA
## 2  2020-08-21  9995.514 -0.0004485525
## 3  2020-08-26  9985.323 -0.0010195666
## 4  2020-08-28  9976.919 -0.0008416865
## 5  2020-09-02  9961.588 -0.0015365857
## 6  2020-09-03  9996.579  0.0035125836
## 7  2020-09-04 10004.710  0.0008132999
## 8  2020-09-08  9976.940 -0.0027756379
## 9  2020-09-09  9997.039  0.0020145025
## 10 2020-09-10  9979.459 -0.0017584791
## 11 2020-09-14  9992.714  0.0013281664
## 12 2020-09-17  9984.878 -0.0007840900
## 13 2020-09-18  9973.713 -0.0011182570
## 14 2020-09-25  9989.647  0.0015976763
## 15 2020-09-28 10005.741  0.0016110549
## 16 2020-10-19 10021.948  0.0016197098
## 17 2020-10-30 10009.792 -0.0012129506
\end{verbatim}

Equity Line:
\includegraphics{exam-paper_files/figure-latex/unnamed-chunk-7-1.pdf}

Orders:

\begin{Shaded}
\begin{Highlighting}[]
\NormalTok{portfolio}\OperatorTok{$}\NormalTok{orders}\OperatorTok{$}\NormalTok{openDate =}\StringTok{ }\KeywordTok{as.Date}\NormalTok{(portfolio}\OperatorTok{$}\NormalTok{orders}\OperatorTok{$}\NormalTok{openDate, }\DataTypeTok{origin=}\StringTok{"1970-01-01"}\NormalTok{)}
\NormalTok{portfolio}\OperatorTok{$}\NormalTok{orders}\OperatorTok{$}\NormalTok{closeDate=}\StringTok{ }\KeywordTok{as.Date}\NormalTok{(portfolio}\OperatorTok{$}\NormalTok{orders}\OperatorTok{$}\NormalTok{closeDate, }\DataTypeTok{origin=}\StringTok{"1970-01-01"}\NormalTok{)}
\NormalTok{portfolio}\OperatorTok{$}\NormalTok{orders}
\end{Highlighting}
\end{Shaded}

\begin{verbatim}
##    openPrice   openDate closePrice  closeDate type riskRewardRatio    amount
## 1    3381.99 2020-08-17    3397.16 2020-08-21 SELL               2 1000.0000
## 2    3443.62 2020-08-25    3478.73 2020-08-26 SELL               2  999.5514
## 3    3478.73 2020-08-26    3508.01 2020-08-28 SELL               2  998.5323
## 4    3526.65 2020-09-01    3580.84 2020-09-02 SELL               2  997.6919
## 5    3580.84 2020-09-02    3455.06 2020-09-03 SELL               2  996.1588
## 6    3455.06 2020-09-03    3426.96 2020-09-04 SELL               2  999.6579
## 7    3426.96 2020-09-04    3331.84 2020-09-08  BUY               2 1000.4710
## 8    3331.84 2020-09-08    3398.96 2020-09-09  BUY               2  997.6940
## 9    3398.96 2020-09-09    3339.19 2020-09-10  BUY               2  999.7039
## 10   3339.19 2020-09-10    3383.54 2020-09-14  BUY               2  997.9459
## 11   3383.54 2020-09-14    3357.01 2020-09-17  BUY               2  999.2714
## 12   3357.01 2020-09-17    3319.47 2020-09-18  BUY               2  998.4878
## 13   3246.59 2020-09-24    3298.46 2020-09-25  BUY               2  997.3713
## 14   3298.46 2020-09-25    3351.60 2020-09-28  BUY               2  998.9647
## 15   3483.34 2020-10-15    3426.92 2020-10-19 SELL               2 1000.5741
## 16   3310.11 2020-10-29    3269.96 2020-10-30  BUY               2 1002.1948
##    status takeProfit stopLoss profitLoss
## 1  CLOSED   3361.698 3392.136  -4.485525
## 2  CLOSED   3422.958 3453.951 -10.191093
## 3  CLOSED   3457.858 3489.166  -8.404512
## 4  CLOSED   3505.490 3537.230 -15.330391
## 5  CLOSED   3559.355 3591.583  34.990913
## 6  CLOSED   3434.330 3465.425   8.130217
## 7  CLOSED   3447.522 3416.679 -27.769451
## 8  CLOSED   3351.831 3321.844  20.098571
## 9  CLOSED   3419.354 3388.763 -17.579583
## 10 CLOSED   3359.225 3329.172  13.254382
## 11 CLOSED   3403.841 3373.389  -7.835187
## 12 CLOSED   3377.152 3346.939 -11.165660
## 13 CLOSED   3266.070 3236.850  15.934765
## 14 CLOSED   3318.251 3288.565  16.093870
## 15 CLOSED   3462.440 3493.790  16.206397
## 16 CLOSED   3329.971 3300.180 -12.156128
\end{verbatim}

Orders Analytics: amountOfBuyOrdersPercentage =
sum(portfolio\(orders\)type==``BUY'')/amountOfOrders
amountOfSellOrdersPercetage =
sum(portfolio\(orders\)type==``SELL'')/amountOfOrders

Missing values in original data:

\begin{Shaded}
\begin{Highlighting}[]
\NormalTok{missingValues}
\end{Highlighting}
\end{Shaded}

\begin{verbatim}
##   Date  Price Change 
##      0      3      0
\end{verbatim}

\hypertarget{code-explainations}{%
\subsection{Code Explainations}\label{code-explainations}}

\hypertarget{a-note-on-strategy-entity-and-rsi-entity}{%
\subsubsection{A note on Strategy Entity and RSI
Entity}\label{a-note-on-strategy-entity-and-rsi-entity}}

It is important to understand the structure and the usefulness of the
Strategy entity. This entity has the purpose, as in normal Trading
systems, of checking for signals and inform about the amount of
data(starting the count from the day before the day considered) needed
by the indicator to compute those signals. In fact, RSI indicator, as
well as most of all Technical indicators, need a certain amount of data
to create a meaningful result and not errors(for instance, an indicator
that expects 14 parameters, if provided with just 6, would throw an
error or an unfaithful result).

\[
RSI Value = 100-\frac{100}{1+\frac{Average Gain}{Average Loss}}
\] Regarding RSI Entity, it's important,when computing its formula, to
take into account the event by which negative or positive returns are
not available for the data (maybe due to a strong uptrend or downtrend).
In order to deal with this problem, the value 0 will be assigned to the
potential void variable.

\begin{Shaded}
\begin{Highlighting}[]
\CommentTok{#[...Code...]}
\ControlFlowTok{if}\NormalTok{ (}\KeywordTok{sum}\NormalTok{(returns}\OperatorTok{>=}\DecValTok{0}\NormalTok{)}\OperatorTok{==}\DecValTok{0}\NormalTok{) \{}
\NormalTok{upReturns =}\StringTok{ }\DecValTok{0}
\NormalTok{\}}
\ControlFlowTok{else}\NormalTok{\{}
\NormalTok{  upReturns =}\StringTok{ }\NormalTok{returns[returns}\OperatorTok{>=}\DecValTok{0}\NormalTok{]}
\NormalTok{\}}
\ControlFlowTok{if}\NormalTok{ (}\KeywordTok{sum}\NormalTok{(returns}\OperatorTok{<}\DecValTok{0}\NormalTok{)}\OperatorTok{==}\DecValTok{0}\NormalTok{)\{}
\NormalTok{ downReturns =}\StringTok{ }\DecValTok{0}
\NormalTok{\}}
\ControlFlowTok{else}\NormalTok{\{}
\NormalTok{  downReturns =}\StringTok{ }\NormalTok{returns[returns}\OperatorTok{<}\DecValTok{0}\NormalTok{];}
\NormalTok{\}}
\CommentTok{#[...Code...]}
\end{Highlighting}
\end{Shaded}

\hypertarget{backtester-logic}{%
\subsubsection{Backtester Logic}\label{backtester-logic}}

\includegraphics{chart of backtester.png} The logic of the backtester is
shown in the chart above. It is, on a basic level, a FOR cycle that runs
from the \texttt{startingPoint} of the strategy. The
\texttt{backtestStrategy(portfolio,\ strategy,\ financialData)} function
is shown below.

\begin{Shaded}
\begin{Highlighting}[]
\NormalTok{startingValue =}\StringTok{ }\NormalTok{strategy}\OperatorTok{$}\KeywordTok{startingPoint}\NormalTok{()}\OperatorTok{+}\DecValTok{1}
\NormalTok{  amountOfDataFromToday =}\StringTok{ }\NormalTok{strategy}\OperatorTok{$}\KeywordTok{amountOfDataFromToday}\NormalTok{()}
\NormalTok{  portfolio}\OperatorTok{$}\NormalTok{strategy =}\StringTok{ }\NormalTok{strategy}

  \ControlFlowTok{for}\NormalTok{ (dateIndex }\ControlFlowTok{in}\NormalTok{ startingValue}\OperatorTok{:}\KeywordTok{nrow}\NormalTok{(financialData)) \{}
\NormalTok{    dataInputForStrategy =}\StringTok{ }\NormalTok{financialData[(dateIndex}\OperatorTok{-}\NormalTok{amountOfDataFromToday)}\OperatorTok{:}\NormalTok{dateIndex}\DecValTok{-1}\NormalTok{,]}
\NormalTok{    todayPrice =}\StringTok{ }\NormalTok{financialData}\OperatorTok{$}\NormalTok{Price[dateIndex]}
\NormalTok{    todayDate =}\StringTok{ }\NormalTok{financialData}\OperatorTok{$}\NormalTok{Date[dateIndex]}
\end{Highlighting}
\end{Shaded}

The backster then will check for orders to close and check whether the
portfolio has still some money inside and if not, it will stop. If it is
the last day of the trading strategy, the OPEN orders will be closed and
the code will stop:

\begin{Shaded}
\begin{Highlighting}[]
\NormalTok{    portfolio}\OperatorTok{$}\KeywordTok{checkForOrdersToClose}\NormalTok{(todayPrice, todayDate)}
\NormalTok{    amountInPortfolio =}\StringTok{ }\NormalTok{portfolio}\OperatorTok{$}\NormalTok{value}

    \ControlFlowTok{if}\NormalTok{ (amountInPortfolio }\OperatorTok{<=}\DecValTok{0}\NormalTok{)\{}
      \ControlFlowTok{break}
\NormalTok{    \}}
    \ControlFlowTok{else} \ControlFlowTok{if}\NormalTok{ (dateIndex }\OperatorTok{==}\StringTok{ }\KeywordTok{length}\NormalTok{(financialData}\OperatorTok{$}\NormalTok{Date))\{}
\NormalTok{     portfolio}\OperatorTok{$}\KeywordTok{closeAllOrders}\NormalTok{(todayPrice, todayDate)}
\NormalTok{    \}}
\end{Highlighting}
\end{Shaded}

Moreover, if it is not the last day of trading and the portfolio has
still a positive value, the Backtester will check for signals. If the
signal is IDLE, which means that the strategy suggests to not trade, the
backtester will move to the next trading day. On the contrary, if there
is a BUY or SELL order, the Backtester will check whether there are open
orders. If there are not open orders, it will create an order and add it
to the orders of the portfolio. If the strategy gives a signal that is
the same of the open order, then the position will be kept in hold (for
instance, the strategy tells to BUY and there is a BUY open order).

\begin{Shaded}
\begin{Highlighting}[]
 \ControlFlowTok{else}\NormalTok{ \{}
      \CommentTok{#'[1) check what strategy says (buy/sell)]}
\NormalTok{      orderType =}\StringTok{ }\NormalTok{strategy}\OperatorTok{$}\KeywordTok{checkForSignals}\NormalTok{(dataInputForStrategy)}
      
      \ControlFlowTok{if}\NormalTok{ (orderType}\OperatorTok{==}\StringTok{"IDLE"}\NormalTok{)\{}
        \ControlFlowTok{next}
\NormalTok{      \}}

      \CommentTok{#'[2) check if there are orders open]}
\NormalTok{      orderResult =}\StringTok{ }\NormalTok{portfolio}\OperatorTok{$}\KeywordTok{checkForOpenOrders}\NormalTok{(orderType)}
\NormalTok{      ordersOpen =}\StringTok{ }\NormalTok{orderResult}\OperatorTok{$}\NormalTok{ordersOpen}
\NormalTok{      orderIndex =}\StringTok{ }\NormalTok{orderResult}\OperatorTok{$}\NormalTok{orderIndex}
      \CommentTok{#'[3) if no order is open, open the order]}
      \ControlFlowTok{if}\NormalTok{ (}\KeywordTok{length}\NormalTok{(ordersOpen)}\OperatorTok{==}\DecValTok{0}\NormalTok{)\{}
\NormalTok{        orderPlaced =}\StringTok{ }\KeywordTok{createOrder}\NormalTok{(}
          \DataTypeTok{openPrice=}\NormalTok{todayPrice, }
          \DataTypeTok{openDate=}\NormalTok{todayDate, }
          \DataTypeTok{type=}\NormalTok{orderType}
\NormalTok{          )}
\NormalTok{        portfolio}\OperatorTok{$}\KeywordTok{addOrder}\NormalTok{(orderPlaced)}
\NormalTok{      \}}
    \CommentTok{#'[4) check if the order is the same of the signal]}
      \ControlFlowTok{else} \ControlFlowTok{if}\NormalTok{ (}\KeywordTok{isOrderOfThisType}\NormalTok{(ordersOpen, orderType))\{ }\CommentTok{# if true}
    \CommentTok{#'[5) hold the order - DO NOTHING]   }
\NormalTok{      \} }
\end{Highlighting}
\end{Shaded}

Lastly, if the order has a type contrary to the one suggested by the
signals, the order will be closed and a new order coherent with the
signal will be opened

\begin{Shaded}
\begin{Highlighting}[]
     \ControlFlowTok{else} \ControlFlowTok{if}\NormalTok{ (}\OperatorTok{!}\KeywordTok{isOrderOfThisType}\NormalTok{(ordersOpen, orderType))\{}
      \CommentTok{#'[7) close the opposite order]}
\NormalTok{      portfolio =}\StringTok{ }\NormalTok{portfolio}\OperatorTok{$}\KeywordTok{closeOrder}\NormalTok{(orderIndex,todayPrice, todayDate)}
      \CommentTok{#'[8) open new order]}
\NormalTok{      orderPlaced =}\StringTok{ }\KeywordTok{createOrder}\NormalTok{(}
        \DataTypeTok{openPrice=}\NormalTok{todayPrice, }
        \DataTypeTok{openDate=}\NormalTok{todayDate, }
        \DataTypeTok{type=}\NormalTok{orderType)}
\NormalTok{      portfolio}\OperatorTok{$}\KeywordTok{addOrder}\NormalTok{(orderPlaced)}
\NormalTok{      \}}
    \ErrorTok{\}}
  \ErrorTok{\}}
\end{Highlighting}
\end{Shaded}

\hypertarget{order-entity-how-an-order-is-opened-and-closed}{%
\subsection{Order Entity: how an order is opened and
closed}\label{order-entity-how-an-order-is-opened-and-closed}}

\hypertarget{scaling-with-new-strategies-and-indicators}{%
\section{Scaling with New Strategies and
Indicators}\label{scaling-with-new-strategies-and-indicators}}

\end{document}
